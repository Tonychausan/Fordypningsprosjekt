\chapter{Introduction}
\label{chap:introduction}

\section{Motivation}
\label{sec:motivation}
People tend to move their hands when they talk, they make gestures.  Gesturing is a known phenomenon, found across cultures, ages, and work. Gestures are even found in individuals that are blind from birth \cite{goldin1999role}. Body movement is a powerful medium for non-verbal interaction \cite{caramiaux2015understanding}. If computers were trained to recognize gestures on top of the traditional User Interface (UI) elements, like text input and speech, we could expand for better expressions and new control alternatives. Speech is a very natural way of communicating, but sound may be inappropriate in certain circumstances that require silence, such as police infiltration operations, or even impossible for the case of deaf people \cite{paudyal2016sceptre}. 

Sign language is a form for human communication based on visual perception. According to World Federation of the Deaf, there are about 70 million deaf people who use sign language as their first language or mother tongue \cite{wfdeaf:sign_language}. Deaf and hard-hearing individuals, who learn the sign language from an early age have to learn both the signed and non-signed varieties that co-exist in the society \cite{bidoli2008english}. The majority of our society do not have a common sign language, and do not have the ability to understand well developed sign languages, such as the American Sign Language (ASL). Communication difficulties between deaf people and hearing people is a consequence of lack of a common language.

An utility device made for translating signed language, can be beneficial for user-to-user communication. Take a scenario where at least one of the involved in the communication wish to communicate using a gesture based form of communication, while the other involved do not have the ability to understand this form of communication. A device capable of this is not only useful for sign language translation, but in every context where gesture based communication is required, such as military communication or other circumstances where sound could be dangerous.


\section{Problem description}
\label{sec:problem_description}
The Myo armband is an off-the-shelf gesture recognition device that uses a set of electromyographic sensors that sense electrical activity in the forearm, combined with a gyroscope, accelerometer and magnetometer to recognize gestures. It can also give haptic feedback vibrations.

\begin{sloppypar}
Electromyography (EMG) is an electrodiagnostic medicine technique for evaluating and recording the electrical activity produced by skeletal muscles. An electromyograph detects the electrical potential generated by muscle cells.
\end{sloppypar}

The goal of this report is not to describe the development of a product, but rather explore capabilities and applications of using electromyography and orientation in gesture control. This report will deal with using the Myo armband to detect muscle gestures with the goal of helping communication of people that rely on arm/hand signals, such as orchestra director, traffic policeman, football referee hand signals, deaf people, military signals, construction site, and hand signals.

\section{Project Scope and Outline}
\label{sec:project_scop_and_outline}
This report will describe an implemented system to recognize gestures using the Myo armband. The system will be used to explore capabilities and applications of utilizing electromyography and IMU, and will not focus on optimizations regarding speed or user experience. 

We will first take a look on related work in \cref{chap:background}. Then move on to a detailed description of the technology provided by the Myo Armband in \cref{chap:myo} and \ref{chap:theory}. In \cref{chap:methodology} we will cover the description of the implemented system. Technical details and mathematical models will also be provided in this chapter. Methods for testing and the received result is propsed in \cref{chap:results}. The last part of the report, \cref{chap:discussion} and \ref{chap:conclusion}, will provide an analysis of the system and the results, and discusses any potential improvements. 
\chapter{Background}
In this chapter we will take a look at other relevant work where the Myo armband or in general EMG have been used for similar or related applications.

\section{EMG-based Controlled Robot}
Various interface systems and prosthetics have been developed to support handicapped people with limit manipulation capability of the upper limb due to traffic accident, cerebral apoplexy, or other afflictions. Many prosthetic arms have been developed for amputees since the 1970’s, and in the paper \cite{fukuda1998emg}, they propose the concept of an EMG-based human-robot interface as rehabilitation aids.

The paper \cite{artemiadis2010emg} propose a methodology for controlling an anthropomorphic robot arm using nine surface EMG electrodes to record the muscular activities. A control interface is proposed, according to which, the user performs motions with his/her upper limb. The recorded electromyographic activity of the muscles can be transformed into kinematic variables that are used to control an anthropomorphic robot arm.

\section{SCEPTRE}
This report is mainly based on a project called SCEPTRE by some researchers from Arizona State University \cite{paudyal2016sceptre}. The overall goal of SCEPTRE is to match gestures.

is to develop a system using two Myo devices to decipher American Sign Language (ASL) gestures, and display the meaning on a smartphone or computer. The SCEPTRE project is an attempt to develop a product toward a system which is ubiquitous, non-invasive, works in real-time, and can be trained interactively by the user. 

The system SCEPTRE 

\section{Alternative Gesture Control Methods}
\subsection{Leap Motion}
\subsection{Kinect}
\subsection{Node (Ring)}
\subsection{Maestro Gesture Glove}

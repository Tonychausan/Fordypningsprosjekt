\section{Electomyography (EMG)}
Electromyography (EMG) is an electrodiagnostic medicine technique to measure muscle response or electrical activity produced by skeletal muscles \cite{wiki:Electromyography}. The nerves control the muscles by electrical signal called impulse, these impulses can be measured and analyzed \cite{WebMD:Electromyogram}. There are different method to measure those signals, but this report will only cover the use surface EMG. Surface EMG is a technique where electrodes are places on the skin overlying a muscle to detect nerve impulses. 

EMG can be used to sense isometric muscular activity which does not translate into movement. This make it possible to detect motionless gestures. One of the main difficulties in analyzing the EMG signal is noisy characteristics. Compared to other biosignals, EMG contains complicated types of noise that are caused by, for example, inherent equipment noise, electromagnetic radiation, motion artifacts, and the interaction of different tissues. Hence, preprocessing is needed to filter out the unwanted noises in EMG \cite{kim2008emg}. Because surface EMG does not get direct measurement of the motor unit activation and many factors can influence the signal, these relations are frequently misinterpreted. Although surface EMG is a useful measure of muscle activation, there are limits to the information that can be extracted from this signal \cite{farina2004extraction}.
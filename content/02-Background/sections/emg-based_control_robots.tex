\section{EMG-based Controlled Robot}
Various interface systems and prosthetics have been developed to support handicapped people with limit manipulation capability of the upper limb due to traffic accident, cerebral apoplexy, or other afflictions. Many prosthetic arms have been developed for amputees since the 1970’s, and in the paper \cite{fukuda1998emg}, they propose the concept of an EMG-based human-robot interface as rehabilitation aids. Another paper \cite{artemiadis2010emg} propose a methodology for controlling an anthropomorphic robot arm using nine surface EMG electrodes to record the muscular activities. A control interface is proposed, according to which, the user performs motions with his/her upper limb. The recorded electromyographic activity of the muscles can be transformed into kinematic variables that are used to control an anthropomorphic robot arm.

This article from CNET \cite{cnet:myoarm} shows a project where amputee Johnny Matheny lost his arm to cancer in 2008. At the Johns Hopkins Applied Physics Laboratory, Matheny worked with a prosthetic arm attached directly to his skeleton, this prosthetic arm is controlled by the use of two Myo armbands on his upper arm that detect the electrical activity of his muscles.
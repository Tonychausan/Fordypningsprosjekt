\chapter{Introduction}
\label{chap:introduction}


\section{Problem description}
\label{sec:problem_description}


\section{Motivation}
\label{sec:motivation}
People move their hands as they talk, they make gestures.  Gesturing is a robust phenomenon, found across cultures, ages, and tasks. Gestureu is even found in individuals blind from birth \cite{goldin1999role}. Body movements are a powerful medium for nonverbal interaction \cite{caramiaux2015understanding}. If computers were trained to recognize gestures on top of the traditional User Interface (UI) elements like text input and speech, we could expand for better expressions and different control alternatives. Speech is a very natural way of communicating, but sound can be inappropriate in certain circumstances that require silence, or even impossible in the case of deaf people \cite{paudyal2016sceptre}. 

Sign language is a form for human communication based on visual perception. According to World Federation of the Deaf, there are about 70 million deaf people who use sign language as their first language or mother tongue \cite{wfdeaf:sign_language}. Deaf and hard-hearing individuals, who learn the sign language from an early age have to learn both the signed and non-signed varieties that co-exist in the society \cite{bidoli2008english}. The majority of our society don't have a common sign language, and do not have the ability to understand developed sign languages, such as the American Sign Language (ASL).

An utility device to help user-to-user communication can be beneficial in a scenario where at least one of the person involved in the communication wish to communicate using a gesture based form of communication, while the other person cannot understand this form of communication. A device capable of this is not only useful for sign languages, but in every context 

\section{Project Scope and Outline}
\label{sec:project_scop_and_outline}
The goal of this project is not to develope a product, but rather explore capabilities and applications of using electromyography and orientation in gesture control.
\chapter{Introduction}
\label{chap:introduction}

\section{Motivation}
\label{sec:motivation}
People move their hands as they talk, they make gestures.  Gesturing is a robust phenomenon, found across cultures, ages, and tasks. Gestureu is even found in individuals blind from birth \cite{goldin1999role}. Body movements are a powerful medium for nonverbal interaction \cite{caramiaux2015understanding}. If computers were trained to recognize gestures on top of the traditional User Interface (UI) elements like text input and speech, we could expand for better expressions and different control alternatives. Speech is a very natural way of communicating, but sound can be inappropriate in certain circumstances that require silence, or even impossible in the case of deaf people \cite{paudyal2016sceptre}. 

Sign language is a form for human communication based on visual perception. According to World Federation of the Deaf, there are about 70 million deaf people who use sign language as their first language or mother tongue \cite{wfdeaf:sign_language}. Deaf and hard-hearing individuals, who learn the sign language from an early age have to learn both the signed and non-signed varieties that co-exist in the society \cite{bidoli2008english}. The majority of our society don't have a common sign language, and do not have the ability to understand developed sign languages, such as the American Sign Language (ASL). Communication between deaf people and hearing people is a consequence of lack of a common language.

An utility device to help user-to-user communication can be beneficial in a scenario where at least one of the person involved in the communication wish to communicate using a gesture based form of communication, while the other person cannot understand this form of communication. A device capable of this is not only useful for sign language translation, but in every context where gesture based communication is required, such as military communication or other circumstances where sound could be dangerous.


\section{Problem description}
\label{sec:problem_description}
The Myo armband is a off-the-shelf gesture recognition device that uses a set of electromyographic sensors that sense electrical activity in the forearm, combined with a gyroscope, accelerometer and magnetometer to recognize gestures. It can also give haptic feedback vibrations.

\begin{sloppypar}
Electromyography (EMG) is an electrodiagnostic medicine technique for evaluating and recording the electrical activity produced by skeletal muscles. An electromyograph detects the electrical potential generated by muscle cells.
\end{sloppypar}

The goal of this report is not to describe the development of a product, but rather explore capabilities and applications of using electromyography and orientation in gesture control. This report will deal with using the Myo armband to detect muscle gestures with the goal of helping communication of people that rely on arm/hand signals, such as orchestra director, traffic policeman, football referee hand signals, deaf people, military signals, construction site, and hand signals.

\section{Project Scope and Outline}
\label{sec:project_scop_and_outline}
In this report
\section{Data Comparison}
\label{sec:data_comparison}
Different comparison methods was used for data comparison. Looking on a graph representation of the data, we can see a visual similarity of the graphs.

\subsection{Dynamic Time Wrapping}
Dynamic time warping (DTW) is a well-known technique to find an optimal
alignment between two given (time-dependent) sequences $x$ and $y$ under certain restrictions \cite{muller2007dynamic}. The goal is to align two sequence of graph by warping the time axis iteratively until an optimal match between the two graph is found.

DTW algorithm work with that it tries to fill a cost matrix. Each element of the cost matrix is the distance of the corresponding elements of the two sequences. Calculating this matrix have a complexity of $\mathcal{O}(n*m)$ where $n$ and $m$ is the size of the sequences $x$ and $y$. 

\subsection{Cross Correlation}
\label{subsec:cross_correlation}
Cross Correlation is a method to estimate the degree of similarity of two series. Cross correlation works on any number of dimension, but for the purpose of this project, we only need to take a look at the one-dimensional cross correlation. Let $x$ and $y$ be two series, then we can define the one-dimensional normalized cross correlation $r$ at delay $d$ as	

\begin{equation}
\label{eq:cross_correlation}
    r(d) = \frac{\sum\limits_{i=0}\limits^{n} [(x(i) - \mean{x}) * (y(i-d) - \mean{y})]}{\sqrt{\sum\limits_{i=0}\limits^{n} (x(i) - \mean{x})^{2}} * \sqrt{\sum\limits_{i=0}\limits^{n} (y(i-d) - \mean{y})^{2}}},
\end{equation}

where $n$ is the number of point and $\mean{x}$ and $\mean{y}$ are the means of the corresponding series \cite{cross_correlation_theory}. The cross correlation have a range of -1 to 1. The value gives information about the series rises and falls relative to each other, where 0 indicating no correlation, $r > 0$ indicating positive correlation and $r < 0$ indicating negative correlation. If both rise at an identical rate, then we have $r = 1$, and opposite $r = -1$ if they fall at an identical rate. 

Given by the formula \ref{eq:cross_correlation} we get an issue when the index is less than 0 or when the index is greater than or equal to the number of points. The most common approach is to ignor this points and let $y(k) = 0$ if $k < 0$ or $k \geq n$ \cite{cross_correlation_code}.